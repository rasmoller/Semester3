\documentclass[../main.tex]{subfiles}
\graphicspath{{fig/}{../fig/}}


\begin{document}

\subsection{Introduktion}
I denne opgave skal vi undersøge, hvordan man med en I2C master kan snakke med temperatur sensoren LM75.
Hertil skal vi bruge viden fra tidligere opgaver og snakke med vores master gennem en UART forbindelse. 
Dette gør vi for at aflæse LM75'erens målinger.

\subsection{Overvejelser}
For at vi kan snakke med LM75'eren er der 2 forbindelser vi skal fokusere på.
\begin{enumerate}
    \item LM75$\iff$PSoC (I2C Master)
    \item PSoC$\iff$PC (UART)
\end{enumerate}
Disse forbindelse vil vi følgende beskrive og kigge nærmere på, hvilke udfordringer der opstår og hvordan vi planlægger at overkomme dem.

\subsubsection{LM75$\iff$PSoC (I2C Master)}
Denne forbindelse foregår via I2C og her skal vi fra PSoC'en sende beskeder som LM75'en modtager og håndtere. Den opbygning vi skal give beskederne kan vi se i datasheetet\footnote{https://www.ti.com/lit/ds/symlink/lm75b.pdf}.
Først skal vi sætte den adresse, som vi skal kommunikere med LM75'en igennem. Dette gør vi med følgende besked struktur.
\pic{LM75I2CAdresse.png}{0.5}{I2C adresse til LM75}{fig:i2cadresse}

\subsubsection{PSoC$\iff$PC (UART)}
For at snakke mellem PSoC'en og vores computer bruger vi et UART komponent. Dette skal sættes op sådan at PSoC'en sender den læste data fra LM75'en til PC'en.
På grund af vi ikke bruger computerens input til noget, har vi ikke noget interrupt på RX benet.

Vores computer sættes op med RealTerm til at modtage fra den USB port som PSoC'en er sat til.
Selve formaterringen af teksten foregår alt sammen på PSoC'en.

Et af problemer i denne forbindelse er, hvordan vi håndtere at sende vores temperatur værdi som en "floating point" værdi.
Vi ved at fra LM75'en får vi to 8-bit heltal (integer) og i dem ligger dataen som på 

\subsection{Implementering}
[TEXT HERE]

\pic{td_i2c}{0.7}{EN TEKST HER TAK}{fig:td}

\pic{design_i2c}{0.7}{ENDNU MERE TEKST HER TAK}{fig:des}

\subsection{Dokumentation}

\pic{temp_read.png}{0.7}{DER SKAL VÆRE TEKSKTST}{fig:temp}

\pic{2temp_read.png}{0.7}{DER SKAL MERE TEKST SE}{fig:temp2}

\pic{i2c_wave}{0.7}{DER SKAL TEKST IGEN}{fig:wave}

\pic{i2c_wave0x49}{0.7}{ENDU MERE TEKST}{fig:wave2}

\subsection{Diskussion}
[TEXT HERE]

\subsection{Konklusion}
[TEXT HERE]

\end{document}