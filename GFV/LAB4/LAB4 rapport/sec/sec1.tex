\documentclass[../main.tex]{subfiles}
\graphicspath{{fig/}{../fig/}}


\begin{document}

\subsection{Introduktion}
Vi vil i denne øvelse vende tilbage til vores tidligere arbejde med at aflæse temperaturen fra en LM75 temperatur sensor fra lab 2. Dog istedet for blot at aflæse temperaturen, ønsker vi nu bruge LM75 sensoren til at måle temperaturen af et varmelegme.

Formålet med denne øvelse vil så bestå i implementeringen af et Control Theory system, som skal gøre det muligt at styre hvordan temperaturen af varmelegemet stiger for at nå en bestemt temperatur. For at kunne gøre dette, vil vi igennem arbejdet med øvelse tilegne os viden om lukkede loop kontrol systemer, og brugen af en PID controller.

\subsection{Overvejelser}

Der vil ikke blev uddybbet videre på LM75 temperatur sensoren i denne rapport, da den ikke er videre relevant for formålet med opgaven.

Dog finder vi det relevant at nævne at der benyttes en I2C forbindelse til kommunikation mellem PSoC ogl PC, og at sensoren inteagere med resten af systemet gennem dens kontakt med varmelegemet.

\subsubsection{Varmelegeme}

%Nævn PWM og hvordan det hænger sammen med modstand og varmelegeme.

\subsubsection{Control Theory}

%Snak om control theory, PID controller

\subsection{Implementering}
%Her forklares koden, og hvordan den er struktureret. Her indsættes også billeder af opstilling, og evt. forklaringer dertil.

\subsection{Dokumentation}
%Her indsættes billeder af dokumentation. Husk billeder at konsol, osciloskop mm.

\subsection{Diskussion}


\subsection{Konklusion}


\end{document}