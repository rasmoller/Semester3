%This is my super simple Real Analysis Homework template

\documentclass{article}
\usepackage[utf8]{inputenc}
\usepackage[english]{babel}
\usepackage{subfiles}
\usepackage{amsthm} %lets us use \begin{proof}
\usepackage{amssymb} %gives us the character \varnothing
\setlength{\parskip = 1em}
\setlength{\parindent = 0em}


\begin{document}
%Front page
\begin{titlepage}
    
    \begin{center}
        \vspace*{1cm}
 
        \Huge
        \textbf{Motor Control}
 
        \vspace{0.5cm}
        \LARGE
        Grænseflader til den Fysiske Verden \\
        \date\today
 
        \vspace{1.5cm}
 
        \textbf{
        Rasmus Møller Nielsen - 201909856 - au633064 \\
	   Christian Bach Johansen - 201709351 - au577526\\
	   Gustav Nørgaard Knudsen - 201807736 - au612485\\
	   Andreas Stavning Erlsev - 201705103 - au}
        
        \vfill
        %\includegraphics[width=0.3\textwidth]{au}
        \vspace{2cm}
 
        Hold nr. 4
 
    \end{center}
\end{titlepage}

\newpage
\newpage

\setcounter{page}{1}

%Start of document

\section{Introduktion}
I følgende journal vil der blive gennemgået øvelsen ``Motor Control'' og delopgaverne vil blive gennemgået hver for sig. Disse delopgaver er valgt at blive opdelt i følgende emner:
\begin{itemize}
\item Think

I ``Think'' vil der blive forklaret hvilke tanker gruppen har gjort sig i forhold til delopgaven og, hvordan teorien der er blevet lært indgår i delopgaven. Derudover vil der væres spørgsmål direkte fra opgaven som vil blive besvaret under ``Reflect'' afsnittet\\
Dette inkludere f.eks.:
\begin{itemize}
\item Multisim opstillinger
\item Pseudo kode
\item Spørgsmål fra delopgaven
\end{itemize}

\item Do

I ``Do'' vil der blive gennemgået opstilling samt dokumentation for opstilling
Dette inkludere f.eks.:
\begin{itemize}
\item Billeder af opstilling
\item Tilslutning af strøm og forbindelser
\item Kode stumper for relevante dele
\end{itemize}

\item Document

Under ``Document'' vil der blive fremvist den data som gruppen får ud af opstillingen.

Dette inkludere f.eks.:
\begin{itemize}
\item Billeder
\item Målinger
\end{itemize}


\item Reflect

I ``Reflect'' vil der blive undret og overvejet om gruppens målinger ser korrekte ud og hvad der kan konkluderes på baggrund af delopgaven

Dette inkludere f.eks.:
\begin{itemize}
\item Svar på spørgsmål
\item Diskussion
\item Konklusion
\end{itemize}

%End of introduction and start of subpages
\end{itemize}

\section{Exercise 1}
\subfile{sec/ex1.tex}

\section{Exercise 2}
\subfile{sec/ex2.tex}

\section{Exercise 3}
\subfile{sec/ex3.tex}

\end{document}
