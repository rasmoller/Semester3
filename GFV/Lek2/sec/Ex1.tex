\documentclass[../main.tex]{subfiles}
\graphicspath{{fig/}{../fig/}}

\begin{document}

\subsection{Introduction}
I denne øvelse skal vi arbejde med PWM kontrol af en DC-motor. Den første del af øvelsen vil her omhandle ændring af rotationshastigheden af DC-motoren, ved ændring af duty-cyclen på et PWM signal.

Formålet med øvelsen er at implementere disse funktioner vha. PSoC microcontroller med tilhørende PSoC Creator software. 

\subsection{Think}
I øvelsen skal vi variere hastigheden på en el-drevet motor, og vi forventer derfor at skulle benytter et PWM-signal til at modificere hastigheden af rotation. Da vi gerne vil ændre duty-cyclen på vores PWM signal i små trin, vil incrementering af en variabel hvor igennem PWM-signalet styres, også blive implementeret.

Via vores PSoC vil vi implementere en UART, der kan give information om den nuværende tilstand af signalet, samt til at modtage ændringer i PWM-signalet.

\subsection{Do}
Som driver for vores DC-motor benytter vi en MOSFET-transistor. Vores PSoC kan kun give en VCC på 5V, og vi benytter derfor en separat strømforsyning til at drive motoren.

    \pic{opstilling1}{0.4}{Implementering med MOSFET}{MOSPimp}

Til programmering af PSoC benyttes PSoC Creator. Øvelsen kræver at vi benytter et PWM signal til at styre DC-motorens hastighed, og bruger UART til at kommunikere ændringer af PWM til PSoC fra konsollen.

Til øvelsen er vi blevet givet en skabelon at arbejde ud fra. Her ser vi i top designet vores UART og PWM blokke som vi skal benytte. PWM-blokken bliver givet et input PWM-clock på 50kHz, og PWM-signalet bliver koblet til en pin, Pin\_PWM. Pin\_PWM bliver så koblet til den fysiske pin 63 eller 2.1, på PSoC. PWM sættes til at have en periode på 100, og med en default CompareValue på 50.

    \pic{Oev1_TD.png}{0.6}{Top design for øvelse 1}{TD11}
    \pic{Oev1_P.png}{0.8}{Pins for øvelse 1}{Pins1D}

UART-blokken tildeles en pin-in, Rx\_1 til signalet PSoC skal modtage fra konsollen, og en pin-out, Tx\_1 til at sende beskeder tilbage til computeren. UART interrupt bliver desuden aktiveret for RX, da vi gerne vil kunne styre PWM-signalet uafhængigt af, hvad PSoC'en laver.

Til faktisk at ændre hastigheden af DC-motoren implementere vi funktionerne ’decreaseSpeed()’ og ’increaseSpeed()’. Begge funktioner indledes med ’UART\_1\_PutString()’, der lader os printe en string til konsollen. Herved før vi en streng tilbage på computeren, så vi kan sikre den korrekte funktion bliver eksekveret.

    \pic{Oev1_CD.png}{0.6}{Implementering af decreaseSpeed() og increaseSpeed()}{Pins1CD}

I ’decreaseSpeed()’ checker vi den nuværende værdi af vores CompareValue på PWM\_1. Hvis værdien ikke er under 0, dercrementeres den med 1. På den måde sænkes tiden hvor transistoren er åben, og den gennemsnitlige spænding til DC-motoren falder. Omvendt checker 'increaseSpeed()' om værdien af compareValue er over 100, ellers incrementerer den compareValue.

Der udnyttes den givne interrupt rutine, til at kalde vores funktioner igennem switch-casen.

\subsection{Document}
Vha. vores Analog Discovery måler vi hvordan PWM signalet ændre sig af at kører ’decreaseSpeed()’ og ’increaseSpeed()’.

    \pic{PWM_mesure}{0.5}{PWM lav Duty-cycle}{PWML}
    \pic{MiddelPWM.png}{0.5}{PWM mellem Duty-cycle}{PWMM}
    \pic{High_PWM}{0.5}{PWM høj Duty-cycle}{PWMH}

Vi ser at ved at kører funktioner kan vi ændre på duty-cyclen af PWM-signalet. Funktionerne ændre signalet i små trin, og vi kan derfor frit stille på hastigheden. Vi ser at PWM-signalet svinger fra 0 til 5V, da PSoC levere en VCC på 5V.

Funktionerne printer desuden de korrekte beskeder i konsollen, så vi har styr over hvlke ændringer der er blevet foretaget til duty-cyclen.

\subsection{Reflect}    
Vi så fra vores målinger af PWM-signalet den opførsel forventet af vores funktioner. Signalet steg og faldte som vi specificerede, og beskederne blev printet rigtigt i konsollen.

Desuden så vi også at vores if-sætninger sikrede at ’WriteCompare()’ ikke sendte ugyldige værdier, som vi ser på graferne da PWM-signalet ikke falder under 0.

\end{document}