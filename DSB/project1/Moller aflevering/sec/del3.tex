\documentclass[../main.tex]{subfiles}
\graphicspath{{fig/}{../fig/}}

%Opgave ekstra opgave
\begin{document}

\section{Opgave 1.15 - Ekko}
I denne sidste opgave handler det om, at skabe et ekko og se hvad der sker når man ligger et ekko oven i det originale ikke forskudte signal.

Først skal man lave et signal.
På Listing \ref{lst:tone} kan man se koden for at lave et signal med en frekvens på 2350 Hz og en amplitude på 3.
\begin{lstlisting}[caption={Oprettelse af ikke forskudt tone}, label={lst:tone}]
Fs = 5000; %Samplingsfrekvens
Ts = 1/Fs; % Tid per sampling
    
length = 1; % Længde i sekunder
Ns = Fs*length; % Antallet af samples
t = [0:Ns-1]*Ts; % Tiden over samplingen

%Tone beskrivelse
f0 = 2350; %2350 Hz tone
A = 3; % Amplitude er 3
tone = A * sin(f0*2*pi*t); %Tonen som funktion    
\end{lstlisting}
Denne del kode vil generere signalet som man kan se på Figure \ref{pic:tone}
\pic{1-15-orig.png}{}{Original tone}{pic:tone}

Efter den er blevet oprettet skal man oprette et ekko, som egentlig bare er den originale tone forskudt. Opgaven specificere, at ekkoet originalt skal forskydes med 150 ms.
\begin{lstlisting}[caption={Oprettelse af ekko}, label={lst:ekko}]
e.delay = 150; %ekko med 150 ms forsinkelse
e.Ns = e.delay * (Fs/1000); %Antallet af samples i ekkoet

tone_ekko = [zeros(1,e.Ns), tone]; %Forsinker ved at smide 0'er ind foran
tone_ext = [tone, zeros(1,e.Ns)]; % Forlænger ved at smide 0'er ind bagved
\end{lstlisting}

I Listing \ref{lst:ekko} kan man se, hvordan vi tilpasser ekkoet og den originale tone så man nemt kan ligge dem oven i hinanden.
På Figure \ref{pic:ekkoorig} kan man se, at ved en frekvens på 2350 Hz med et ekko på 150 ms får et signal som går ud med sig selv i det tidsrum, hvor de krydser.
Dette resulterer i, at man hører to korte bip med et mellemrum imellem. 
Herimod, hvis man ændrer ekkoet´s forskydning får man en kontinuerlig lyd, som ændres i styrke alt efter om tidsrummet er efter ekkoet er startet og før det originale er slut.
Dette skyldes, at der enten sker positiv interferens (ved 300ms ekko på Figure \ref{pic:ekko300}) eller negativ interferens (ved 150ms på Figure \ref{pic:ekkoorig}).
\pic{1-15-ekko+orig.png}{}{Ekko og original tone lagt sammen}{pic:ekkoorig}
\pic{1-15-300.png}{}{Ekko på 300ms}{pic:ekko300}




\end{document}