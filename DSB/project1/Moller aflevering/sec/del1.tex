\documentclass[../main.tex]{subfiles}
\graphicspath{{fig/}{../fig/}}

%Opgave 1-3
\begin{document}

\section{Opgave 1 - Find antal samples}
Som udgangspunkt har vi fået udleveret 3 filer, hvor 2 af dem ($S_1$ og $S_3$) er mono imens at $S_2$ er stereo.
Disse filer skal indlæses i matlab og gemmes som en lang række af samples.

Dette gør man ved hjælp af følgende kode:

\begin{lstlisting}[caption={Indlæsning af samples fra fil}, label={lst:sampleRead}]
% Loading files into arrays of samples
[y(1).sample, ~] = audioread('Signal_s1.wav');
[y(2).sample, ~] = audioread('Signal_s2.wav');
[y(4).sample, Fs] = audioread('Signal_s3.wav');
\end{lstlisting}
    
Som vi kan se i Listing \ref{lst:sampleRead} bbliver filerne indlæst med functionen \textit{audioread('filename')}.
Funktionen returnerer her en array af samples og den frekvens der er optaget med.
Vi får at vide at alle frekvenserne er 44.100 Hz.

For herefter at finde ud af hvor mange samples der er per signal kan man bruge funktionen \textit{length()} til at finde antallet af samples.
(Ellers vil man også kunne aflæse dem ude i ens workspace). \\
Længden på signalerne bliver:
\[length(S_1) = 4213759\]   
\[length(S_2) = 8753617\]
\[length(S_3) = 1270957\]
\[Fs = 44.100 Hz\]

Antallet af samples vil senere blive refereret til som nS (number of Samples).

\section{Opgave 2 - Plot signaler}
For at plotte signalerne bruger man funktionen \textit{plot(x, y)}.
Hertil kan man bruge funktionerne \textit{xlabel("label")}, \textit{ylabel("label")} og \textit{title("title")} til at tilføjer titler til akserne og hele plottet.

Da jeg opbevarer mine signaler og deres værdier i en struct bruger jeg en for løkke til at tilgå dem.
Før jeg kan plotte skal jeg udregne tiden i sekunder som signalet løber over. Dertil bruger jeg formlen:
\[t = nS\cdot \frac{1}{Fs}\]
Her er \textit{Fs} frekvensen for sampling. Så $\frac{1}{Fs}$ vil være tiden per sampel og ved at gange det med antallet af sampels får man tiden for hele signalet.

\pic{fig/plots1-3.png}{1}{Plot af signaler med korrekte akser og titel}{pic:plot} 
På Figure \ref{pic:plot} kan man se begge kanaler af $S_2$ samt $S_1$ og $S_3$ alle med Tiden \textit{t} på X-aksen, Amplitude på Y-aksen og en passende titel til hvad det viser.


\section{Opgave 3 - Find værdier for signalerne}
I det her afsnit skal jeg finde følgende værdier for signalerne
\begin{itemize}
    \item max
    \item min
    \item mean (gennemsnit)
    \item rms
    \item effekt
\end{itemize}

Til de første 4 vil jeg bruge følgende funktioner:
\begin{itemize}
    \item \textit{max()}
    \item \textit{min()}
    \item \textit{mean()}
    \item \textit{rms()}
\end{itemize}

og for at udregne effekten vil jeg bruge følgende formel:
\[E = \sum {A_{samples}}^2\]

Efter at have kørt funktionerne og brugt formlen til udregning af effekt får jeg følgende værdier:
\begin{itemize}
    \item $S_1$
    \begin{itemize}
        \item max = $0.9306$
        \item min = $-0.8205$
        \item mean = $-4.28 \cdot 10^{-5}$
        \item rms = $0.1199$
        \item effekt = $6.0558 \cdot 10^4$
    \end{itemize}
    \item $S_2Venstre$
    \begin{itemize}
        \item max = $0.9558$
        \item min = $-0.9558$
        \item mean = $-1.94 \cdot 10^{-5}$
        \item rms = $0.3383$
        \item effekt = $1.0018 \cdot 10^6$
    \end{itemize}
    \item $S_2Højre$
    \begin{itemize}
        \item max = $0.9558$
        \item min = $-0.9558$
        \item mean = $-1.94 \cdot 10^{-5}$
        \item rms = $0.3548$
        \item effekt = $1.1022 \cdot 10^6$
    \end{itemize}
    \item $S_3$
    \begin{itemize}
        \item max = $0.9600$
        \item min = $-0.9192$
        \item mean = $-5.74 \cdot 10^{-4}$
        \item rms = $0.0919$
        \item effekt = $1.0739 \cdot 10^4$
    \end{itemize}
\end{itemize}

\section{Opgave 4 - Find crest faktorer og sammenlign}

Til udregning af Crest faktoren bruger vi følgende formel:
\[
    C = 20 \cdot log_{10}(\frac{max}{rms})
\]

\[C_{S_1} = 17.8006\] 
\[C_{S_2venstre} = 9.0213\]
\[C_{S_2højre} = 8.6064\]
\[C_{S_3} = 20.3770\]




\end{document}

