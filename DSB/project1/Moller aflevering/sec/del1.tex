\documentclass[../main.tex]{subfiles}
\graphicspath{{fig/}{../fig/}}

%Opgave 1-3
\begin{document}

\section{Opgave 1 - Find antal samples}
Som udgangspunkt har vi fået udleveret 3 filer, hvor 2 af dem ($S_1$ og $S_3$) er mono imens at $S_2$ er stereo.
Disse filer skal indlæses i matlab og gemmes som en lang række af samples.

Dette gør man ved hjælp af følgende kode:

\begin{lstlisting}[caption={Indlæsning af samples fra fil}, label={lst:sampleRead}]
% Loading files into arrays of samples
[y(1).sample, ~] = audioread('Signal_s1.wav');
[y(2).sample, ~] = audioread('Signal_s2.wav');
[y(4).sample, Fs] = audioread('Signal_s3.wav');
\end{lstlisting}
    
Som vi kan se i Listing \ref{lst:sampleRead} bbliver filerne indlæst med functionen \textit{audioread('filename')}.
Funktionen returnerer her en array af samples og den frekvens der er optaget med.
Vi får at vide at alle frekvenserne er 44.100 Hz.

For herefter at finde ud af hvor mange samples der er per signal kan man bruge funktionen \textit{length()} til at finde antallet af samples.
(Ellers vil man også kunne aflæse dem ude i ens workspace). \\
Længden på signalerne bliver:
\[length(S_1) = 4213759\]   
\[length(S_2) = 8753617\]
\[length(S_3) = 1270957\]

Antallet af samples vil senere blive refereret til som nS (number of Samples).

\section{Opgave 2 - Plot signaler}
For at plotte signalerne bruger man funktionen \textit{plot(x, y)}.
Hertil kan man bruge funktionerne \textit{xlabel("label")}, \textit{ylabel("label")} og \textit{title("title")} til at tilføjer titler til akserne og hele plottet.

Da jeg opbevarer mine signaler og deres værdier i en struct bruger jeg en for løkke til at tilgå dem.

\pic{"fig/t.txt"}{}{Plot af signaler med korrekte akser og titel}{pic:plot}
På billede \ref{pic:plot}


\section{Opgave 3 - Find værdier for signalerne}



\section{Opgave 4 - Find crest faktorer og sammenlign}




\end{document}

