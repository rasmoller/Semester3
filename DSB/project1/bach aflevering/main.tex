\documentclass{article}
\usepackage[utf8]{inputenc}
%\usepackage[english]{babel}
\usepackage{subfiles}
\usepackage{amsthm} %lets us use \begin{proof}
\usepackage{amssymb} %gives us the character \varnothing
\usepackage[top=3cm, bottom=3cm, left=2.5cm, right=2.5cm]{geometry}
\usepackage{graphicx}
\usepackage[compact]{titlesec}
\usepackage{float}
\usepackage{listings}
\usepackage[labelfont=bf,listformat=simple]{caption}
\usepackage{array}
\usepackage{enumitem}
\usepackage{float}

% Standard lst
\lstset{
numbers=left, 
numberstyle=\small, 
numbersep=8pt, 
frame = single, 
language=matlab, 
framexleftmargin=15pt,
extendedchars=true,
literate = {æ}{{\ae}}1 {å}{{\r a}}1 {ø}{{\o}}1 {Ø}{{\O}}1 {Æ}{{\AE}}1 {Å}{{\r A}}1
}

% Presets
\graphicspath{{fig/}}
\pagestyle{plain}
\setlength{\parskip}{1em}
\setlength{\parindent}{0em}

% =============================
%           Commands
% =============================

% Easy pictures
% \pic{Picture path}{picture size as page size}{Caption}{Label}
\newcommand{\pic}[4]{
	\begin{figure}[H]
		\centering
		\includegraphics[width=#2\textwidth]{#1}
		\caption{#3}
		\label{#4}
	\end{figure}
	\hspace*{\fill}
}

% Credits to: 
% https://tex.stackexchange.com/questions/10293/latex-template-for-use-cases
\newcommand\tabularhead[1]{
\begin{table}[h]
  \begin{tabular}{|p{0.25\linewidth}|p{0.6\linewidth}|}
    \hline
    \textbf{Navn} & \textbf{#1} \\
    \hline}

  \newcommand\addrow[2]{#1 &#2\\ \hline}

  \newcommand\addmulrow[2]{ \begin{minipage}[t][][t]{2.5cm}#1\end{minipage}% 
     &\begin{minipage}[t][][t]{8cm}
      \begin{enumerate} #2   \end{enumerate}
      \end{minipage}\\ \hline} 

  \newenvironment{usecase}{\tabularhead}
{\hline\end{tabular}\end{table}}

% ===========================================
%                DOCUMENT
% ===========================================

\begin{document}

% ===========================================
%                Front page
% ===========================================

\begin{titlepage}
    
    \begin{center}
        \vspace*{1cm}
 
        \Huge
        \textbf{I3DSB - Mini projekt 1}
 
        \vspace{0.5cm}
        \huge
        Digitale bølger \\
        \date\today
 
        \vspace{1.5cm}
 
        \large
        \begin{tabular}{c|lr}
	    SW & Christian Bach Johansen & au577526\\
        \end{tabular}
        
        
        \vfill
        % \includegraphics[width=0.3\textwidth]{au2}
        \vspace{2cm}
 
    \end{center}
\end{titlepage}

\newpage
\newpage

\setcounter{page}{1}

% ===========================================
%            Start of pages
% ===========================================

\section{Introduktion}

I dette mini projekt vil jeg arbejde med sampling og analyse af digital lyd-signaler. Øvelsen vil generelt omhandle behandlingen af tre udleverede lyd-signaler, hvorpå 9 forskellige øvelser vil blive foretaget. Projektet vil desuden indeholde arbejde med funktioner i Mattab, som f.eks. plotning af data og udregninger på arrays.

\section{Delopgave 1 \& 2}

\subfile{sec/del1.tex}

\section{Delopgave 3 \& 4}

\subfile{sec/del2.tex}

\section{Delopgave 5 \& 6}

\subfile{sec/del3.tex}

\section{Delopgave 7 \& 8}

\subfile{sec/del4.tex}

\section{Delopgave 9}

\subfile{sec/del5.tex}

\section{opgave 1.15 fra DSB lektion 3}

\subfile{sec/del6.tex}

\end{document}