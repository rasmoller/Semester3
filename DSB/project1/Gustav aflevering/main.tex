\documentclass{article}
\usepackage[utf8]{inputenc}
%\usepackage[english]{babel}
\usepackage{amsthm} %lets us use \begin{proof}
\usepackage{amssymb} %gives us the character \varnothing
\usepackage[top=3cm, bottom=3cm, left=2.5cm, right=2.5cm]{geometry}
\usepackage{graphicx}
\usepackage[compact]{titlesec}
\usepackage{float}
\usepackage{listings}
\usepackage[labelfont=bf,listformat=simple]{caption}
\usepackage{array}
\usepackage{enumitem}
\usepackage{float}
\usepackage{amsmath}
\usepackage{xr} 

\usepackage{subfiles}

% Standard lst
\lstset{
  numbers=left, 
  numberstyle=\small, 
  numbersep=8pt, 
  frame = single, 
  language=matlab, 
  framexleftmargin=15pt,
  extendedchars=true,
  tabsize=2,
  literate = {æ}{{\ae}}1 {ø}{{\o}}1 {å}{{\r a}}1 {Æ}{{\AE}}1 {Ø}{{\O}}1 {Å}{{\r A}}1,
  xleftmargin=2em,
}

% Presets
\graphicspath{{fig/}}
\pagestyle{plain}
\setlength{\parskip}{1em}
\setlength{\parindent}{0em}

% =============================
%           Commands
% =============================

% Easy pictures
% \pic{Picture path}{picture size as page size}{Caption}{Label}
\newcommand{\pic}[4]{
	\begin{figure}[H]
		\centering
		\includegraphics[width=#2\textwidth]{#1}
		\caption{#3}
		\label{#4}
	\end{figure}
	\hspace*{\fill}
}

% ===========================================
%                DOCUMENT
% ===========================================

\begin{document}

% ===========================================
%                Front page
% ===========================================

\begin{titlepage}
    
  \begin{center}
    \vspace*{1cm}

    \Huge
    \textbf{I3DSB - Mini projekt A}

    \vspace{0.5cm}
    \huge
    Sampling og analyse af digital musik- og talesignaler \\
    \date\today

    \vspace{3cm}

    \Large
    SW - Gustav Nørgaard Knudsen - au612485 
    
    \vfill
    \includegraphics[width=0.3\textwidth]{au2}
    \vspace{2cm}

    Hold nr. 13

  \end{center}
\end{titlepage}

\newpage
\newpage

\setcounter{page}{1}

% ===========================================
%            Start of pages
% ===========================================

\section*{Introduktion}
Formålet med dette miniprojekt er at give en forståelse for hvordan MATLAB håndterer digitale signaler såsom lyd og hvordan disse maniuleres. Lyd opbevares i matricer i MATLAB, som så kan plottes til deres tidsenhed, for at skabe plots af lydsignalet. Desuden er det også muligt at høre lyde, f.eks. med funktioner som soundsc(). Før jeg kan begynde på opgaverne, starter jeg med at indlæse mine .wav filer med audioread(). Jeg opretter en struct, der kan indeholde alt information om mine forskellige lydfiler, samt får opdelt s2 op i left og right dele. Dette kan ses på listing \ref{lst:start}. Desuden nulstiller jeg alle tidligere defineret variabler med med clear og clc, for ikke at få overlap fra tidligere.

\begin{lstlisting}[caption=Setup af signalerne i MATLAB som structs, label=lst:start]
clear;
clc;

% Define samples
s1 = 1;
s2left = 2;
s2right = 3;
s3 = 4;

% Loading files into arrays of samples
[y(s1).sample, ~] = audioread('Signal_s1.wav');
[y(s2left).sample, ~] = audioread('Signal_s2.wav');
[y(s3).sample, Fs] = audioread('Signal_s3.wav');

% Splits signal 2 into its own channels
y(s2right).sample = y(s2left).sample(:,2);
y(s2left).sample = y(s2left).sample(:,1);

% Name samples
y(s1).name = "Signal\_s1.wav";
y(s2left).name = "Signal\_s2.wav - channel 1";
y(s2right).name = "Signal\_s2.wav - channel 2";
y(s3).name = "Signal\_s3.wav";
\end{lstlisting}

For at gøre nogle af de kommende øvelser nemmere at skrive og forstå, er har jeg oprettet et for-loop, der kan løbe igennem alle structs, og oprette de variabler og matricer der kan blive nødvendige. 

\begin{lstlisting}[caption={Oprettelse af varaibler til alle structs}]
% Predefine variables for s1, s2left, s2right, s3 and s1reS
for i = 1:length(y)
  y(i).sample = y(i).sample';
  % number of Samples 1, 2, 3
  y(i).nS = length(y(i).sample);
  %Calculating x axis for signals
  y(i).time = [0:y(i).nS-1]*(1/Fs);
  %Find max, min, mean, rms & effect
  y(i).max = max(y(i).sample);
  y(i).min = min(y(i).sample);
  y(i).mean = mean(y(i).sample);
  y(i).rms = rms(y(i).sample);
  y(i).effect = sum(y(i).sample.^2);
  %Calculating Crest value
  y(i).crest = 20*log10(y(i).max/y(i).rms);
end
\end{lstlisting}

\subfile{sec/del1.tex}

\subfile{sec/del2.tex}

\subfile{sec/del3.tex}

\end{document}