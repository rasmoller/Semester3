\documentclass[../main.tex]{subfiles}
\graphicspath{{fig/}{../fig/}}

%opgave 9 + ekstra opgave
\begin{document}

\section{Opgave 9}

For at få en eksponentielt aftagning på s3, kræver det at s3 ganges med en "envelope" der kan fungere som der ønskes. Her at der på den sidste \( 1 /3 \) af s3, skal aftages eksponentielt, ned til en amplitude på \( 5\% \) af den originale amplitude. Dette kan opnås med funktionen for en eksponentielt aftagende funktion, da vi kender vores skæring med y-aksen, og 2 punkter på grafen, skæringen med y-aksen i (0,1) samt slutpunktet, der skal være (\(\frac{1}{3}time, 0.05\)). Funktionen kan ses her under i ligning \ref{mat:eks}. 

\begin{gather}
  y(x) = b \cdot a^x \label{mat:eks}\\
  b = \text{y-værdi foor skæring med y-aksen} \\
  a = \left(\dfrac{y_2}{y_1}\right)^{ \dfrac{1}{x_2-x_1}} \text{, hvor} (x_1,y_1) \text{ og }  (x_2,y_2)  \text{ er punkter på grafen}
\end{gather}

\begin{lstlisting}[caption={Fade out af s3}label=lst:s3fade]
%% Exercise 9
% Tage sidste tredjedel af s3
t = y(s3).time(1:y(s3).nS/3+1);
% Eksponentielt aftagende: b*a^x
% b = Skæring med y i (0,1)
% a = (y2/y1)^1/(x2-x1)
a = (0.05/1)^(1/(t(end)-t(1)));
% Matrice der er eksponentielt aftagenende
% Starter i 1 og går til 0.05 ved 1/3 af time
e = 1*a.^t;

% s3fade
s3fade = 7;
y(s3fade).sample = y(s3).sample;
y(s3fade).time = y(s3).time;
% Påfør e på sidste 1/3 af samples
y(s3fade).sample(length(y(s3fade).sample)*2/3:end) 
  = e.*y(s3fade).sample(length(y(s3fade).sample)*2/3:end);

figure
plot(y(s3fade).time, y(s3fade).sample, Fs)
%soundsc(y(s3fade).sample)
\end{lstlisting}

Som det kan ses på figur \ref{fig:s3fade}, er den ønskede effekt opnået. På den sidste 1/3 af signalet, bliver stemmen dæmpet eksponentielt. Ved at lytte til signalt kan det også høre, den ønskede effekt er opnået.

\pic{s3_fade}{1}{s3 med fade effekt}{fig:s3fade}



\section{Opgave 1.15}




\end{document}