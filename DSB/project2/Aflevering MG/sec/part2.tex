\documentclass[../main.tex]{subfiles}
\graphicspath{{fig/}{../fig/}}

% Del 3 og 4

\begin{document}

\section{Opgave 3}
I opgave 3 skal vi bestemme effekten for de lave frekvenser (Under 80Hz) og effekten for de høje frekvenser (over 80 Hz).
Dette skal vi gøre for begge de signaler vi har brugt i opgave 1 og 2 (Vindmølle støj og computer blæser støj).
Til dette bruger vi følgende metode i matlab:

\begin{lstlisting}[caption=Udregning af frekvens effekter, label=lst:freqEffect]
%Opstiller frekvenser over 80 H< og under 80Hz
    sound(i).fft_sort_high = sound(i).sample_fft(81:end);
    sound(i).fft_sort_low = sound(i).sample_fft(1:80);

%Bestemmer effekten at de to intervaller
    sound(i).effect_high = sum(sound(i).fft_sort_high.^2);
    sound(i).effect_low = sum(sound(i).fft_sort_low.^2);
\end{lstlisting}

Hertil skal vi også udregne effektforholdet mellem høj- og lav-frekvenseffekterne. 
Dette gøres på følgende måde:
\[
    \frac{P_{low}}{P_{high}}
\]
Følgende er koden i matlab brugt til at udregne effetkforholdet.
\begin{lstlisting}[caption=Udregning af effektforhold, label=lst:effectRel]
% Calculate the effekt relation
sound(i).effect_rel = abs(sound(i).effect_low)/abs(sound(i).effect_high);
\end{lstlisting}

Efter at disse 2 ovenstående stykker kode er kørt kan vi gå ind og kigge på de effekt forhold vi har fået.

\begin{table}[h]
    \centering
    \resizebox{\textwidth}{!}{%
    \begin{tabular}{|l|lll|}
    \hline
    Lydnavn                                     & \multicolumn{1}{c|}{$P_{Low}$}       & \multicolumn{1}{c|}{$P_{High}$}      & $P_{Rel}$        \\ \hline
    Wind turbine sound - high quality audio.mp3 & 159.06 & 165.56 & 0.96 \\ \hline
    Computer Fan - Sound Effect.mp3             & 91774990.50  & 91779089.50 & 0.99 \\ \hline
    \end{tabular}%
    }
    \caption{Tabel over frekvenseffekter og effektforhold}
    \label{tab:effectTabel}
\end{table}

På Table \ref{tab:effectTabel} kan man se en oversigt over både frekvens effekterne og effekt forholdene.
Her kan man se at "Computer Fan - Sound Effect.mp3" har den højeste effekt både ved de lave og høje frekvenser. 
Grunden til at de er så meget større er angiveligt lydfilen, som har en meget højere volumen, da den f.eks. kunne ha været optaget tættere på.
Herudover kan man se, at der er mere energi i de høje frekvenser i begge signaler, dog er forholdet lidt større i vindmølle signalet, men dette kan være fordi effekterne er meget højere i blæser signalet.

\section{Opgave 4}
I denne delopgave skal vi udregne frekvensopløsningen for de 2 støj signaler.

\begin{lstlisting}[caption=Udregning af effektforhold, label=lst:freqRes]
% Calculate the frequency resolution
sound(i).freq_res = sound(i).freq_sample/length(sound(i).samples);
\end{lstlisting}

På listing \ref{lst:freqRes} kan man se at vi bruger nedenstående formel til udregningen af frekvensopløsningen.
\[res = \frac{f_{samples}}{N}\]

Ved bagefter så at aflæse værdierne fra matlab for man at svarene bliver som følgende:

\begin{table}[h]
    \centering
    \begin{tabular}{|l|c|}
    \hline
    Navn & Frekvensopløsningen \\ 
    \hline
    Wind turbine sound - high quality audio.mp3 & 0.00161 Hz \\
    \hline
    Computer Fan - Sound Effect.mp3 & 0.00748 Hz \\
    \hline
    \end{tabular}
    \caption{Tabel over frekvensopløsninger}
    \label{tab:freqResTable}
\end{table}

Resultatet vi får er, antallet af Hertz per sample. Så for "Computer Fan - Sound Effect.mp3" får man at der på 1 sample er gået 0.00748 Hz. Dette er en tilpas høj opløsning, ellers kunne man have tilføjet ekstra samples for at øge opløsningen.

\end{document}