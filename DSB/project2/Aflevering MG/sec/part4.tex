\documentclass[../main.tex]{subfiles}
\graphicspath{{fig/}{../fig/}}

% Opgave 12.12 og 12.15

\begin{document}

\section{Opgave 3.12}
Opgave 3.12 fra Lyons lyder som følgende:
\begin{quote}
    Forestil en N størrelse x(n) tids sekvens, som DFT er beskrevet af X(m), hvor $0 <= m <= N-1$. 
    Stående i denne situation mener en hjemmeside på internettet at "Summen af alle X(m) værdier er lig N gange den første værdi i x(n)." 
    Undersøg om dette er sandt eller falskt.
\end{quote}

Dette vil vi teste ved at først generere nogle tilfældige signaler og herefter dem på følgende ligning:
\[
    \sum X(m) \approx N \cdot x(1)    
\]

\begin{lstlisting}[caption=MatLab kode til udregning af problematikken fra 3.12, label=lst:Opg3.12]
for i = 1:2
    %y(i).samples = (2*rand(5000,1))-1;
    y(i).length = length(y(i).samples);
    
    y(i).dft = fft(y(i).samples, y(i).length-1);
    
    abs(sum(y(i).dft))
    abs(y(i).length*y(i).samples(1))
end    
\end{lstlisting}

I koden på listing \ref{lst:Opg3.12} løber vi en forløkke igennem med længden 2, for at sikre os at vi har flere resultater, der viser det samme resultat.
Først får vi DFT'en ved hjælp af \textit{fft()} funktionen og herefter laver vi udregninger for summen af DFT'en og N gange med den første værdi i x(n).

\section{Opgave 3.15}
I opgave 3.15 er der givet et signal med en længde på 902 samples (x(n)). Denne sekvens er lyden af "A3" ("A"(220 Hz) tonen lige under det midterste "C"(262 Hz)) fra en akustisk guitar. 
Guitaren er samplet ved en samplingfrekvens på 22,255 kHz.

\pic{opg3.15a.png}{0.9}{Plot af x(n)}{fig:plot3.15}
På ovenstående billede kan man se x(n), som var guitaren der spillede en "A3" tone.

Herefter laver vi en frekvensanalyse på signalet og finder DFT'en.

\pic{opg3.15b.png}{0.9}{}{fig:DFT3.15}
På Billede \ref{fig:DFT3.15} kan man se X(m), fourier transformationen af x(n).

\subsection{a)}
I første del af opgaven skal vi finde den fundamentale frekvens for signalet.
Til dette kan vi bruge frekvens bins til at bestemme i hvilket interval grundfrekvensen befinder sig.

\[
    \frac{f_s}{N} = \Delta f
\]
\[
    \frac{22255 Hz}{902} = 24.67 Hz
\]

Plottet fra Figure \ref{fig:DFT3.15} viser, hvor den fundamentale frekvens ligger, ved aflæsning kan man se, at den ligger i 9'ende bin.

Derfor ganger vi vores $\Delta f$ med 9 for at finde, hvor denne frekvens bin slutter.
\[
    24.67 Hz \cdot 9 = 222.06 Hz
\]
Denne bin ved vi så har en slutning på 222,06 Hz og for at finde starten trækker vi 24,67 Hz fra igen.
\[
    222.06 Hz - 24.67 Hz = 197.38 Hz
\]

Herefter kan vi stille os selv spørgsmålet om vores A3 tone er indenfor disse rammer.
\[
    197.38 Hz < 220 Hz < 222.06 Hz
\]
Fra ovenstående udregning kan vi konkludere, at A3 tonen ligger i denne frekvens bin.

\subsection{b)}
\pic{opg3.15c.png}{0.9}{}{fig:logDFT3.15}
Næste spørgsmål går ud på, at vi skal finde den højeste ikke nul frekvens bin.
Ved at aflæse på Figure \ref{fig:logDFT3.15} kan man se, at vores 9'ende bin ligger på 0 dB. Hertil kan man se, at den næsthøjeste
frekvens bin med et spectral komponent er bin 18. Denne bin ligger i følgende interval: 
\[[419.39 , 444.06[\] 
Denne udregning er gjort med samme metode som i forrige del af 3.15.
Vi ved, at anden harmonik vil ligge ved den dobbelte frekvens, som ville være 440 Hz og det passer med at den ligger indenfor rammerne af vores interval.

\end{document}