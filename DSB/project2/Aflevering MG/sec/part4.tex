\documentclass[../main.tex]{subfiles}
\graphicspath{{fig/}{../fig/}}

% Opgave 12.12 og 12.15

\begin{document}

\section{Opgave 3.12}
Opgave 3.12 fra Lyons lyder som følgende:
\begin{quote}
    Forestil en N størrelse x(n) tids sekvens, som DFT er beskrevet af X(m), hvor $0 <= m <= N-1$. 
    Stående i denne situation mener en hjemmeside på internettet at "Summen af alle X(m) værdier er lig N gange den første værdi i x(n)." 
    Undersøg om dette er sandt eller falskt.
\end{quote}

Dette vil vi teste ved at først generere nogle tilfældige signaler og herefter dem på følgende ligning:
\[
    \sum X(m) \approx N \cdot x(1)    
\]

\begin{lstlisting}[caption=MatLab kode til udregning af problematikken fra 3.12, label=lst:Opg3.12]
for i = 1:2
    %y(i).samples = (2*rand(5000,1))-1;
    y(i).length = length(y(i).samples);
    
    y(i).dft = fft(y(i).samples, y(i).length-1);
    
    abs(sum(y(i).dft))
    abs(y(i).length*y(i).samples(1))
end    
\end{lstlisting}

I koden på listing \ref{lst:Opg3.12} løber vi en forløkke igennem med længden 2, for at sikre os at vi har flere resultater, der viser det samme resultat.
Først får vi DFT'en ved hjælp af \textit{fft()} funktionen og herefter laver vi udregninger for summen af DFT'en og N gange med den første værdi i x(n).


\section{Opgave 3.15}

\end{document}